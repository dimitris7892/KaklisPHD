
            \documentclass[a4paper,12pt,dvipsnames]{scrartcl}
            
            
            \usepackage[
                pdftitle={},
                pdfsubject={},
                pdfauthor={},
                pdfkeywords={},
                pdftex=true,
                colorlinks=true,
                breaklinks=true,
                citecolor=black,
                linkcolor=black,
                menucolor=black,
                urlcolor=black
            ]{hyperref}
            
            
            \usepackage[utf8]{inputenc}
            \usepackage[T1]{fontenc}   			
            
            
            \usepackage{subfigure}
            \usepackage[subfigure]{tocloft}
            \usepackage{hyperref}
            \usepackage{pdfcomment}
            \usepackage{todonotes}
            \usepackage{mdwlist} 
            
           
            \usepackage[normalem]{ulem}
            \usepackage[autostyle=true,english=american]{csquotes} 
            \renewcommand{\mkblockquote}[4]{\enquote{#1}#2\ifterm{\relax}{#3}#4} 
            \renewcommand{\mkcitation}[1]{#1} 
            \usepackage{ragged2e} 
            
            \interfootnotelinepenalty10000  
            
            
            \usepackage{setspace} % Package zur Veränderung des Zeilenabstandes innerhalb des Textes
            \expandafter\def\expandafter\quotation\expandafter{\quote\singlespacing} % Einfacher Zeilenabstand in langen Zitaten
            \onehalfspacing % 1,5-facher Zeilenabstand im gesamten Dokument
            %\renewcommand{\baselinestretch}{1.5} % Zeilenabstand 
            %\setlength{\footnotesep}{10.0pt} % Abstand zwischen Fußnote und Text 
            
            % Schusterjungen und Hurenkinder vermeiden
            \clubpenalty = 10000
            \widowpenalty = 10000 
            \displaywidowpenalty = 10000
            
            % Mathe-Zeugs
            \usepackage{amsmath,amsfonts,amssymb} % Package für mathematische Umgebungen (wie Formelgestaltungen, etc.) 
            \usepackage{array} % Package zur Erzeugung von Matrizen (legt Position und Spalten fest)
            \usepackage{textcomp} % Package zur Generierung zusätzlicher Symbolzeichen 
            
            % Farben
            \usepackage{xcolor}
            
            % Grafiken
            \usepackage{graphicx} % Grafiken einfügen
            \usepackage{svg} % .svg's einbinden
            \usepackage{float} % Package zur Festlegung der Position von Tabellen/Abbildungen
            \usepackage[clockwise]{rotating} % Drehung von Bildern & Tabellen
            
            % TikZ
            \usepackage{tikz}
            \usepackage{subfigure}
            
            % Packages für das Erstellen von Tabellen
            \usepackage{tabularx} % Package zur Gestaltung von Tabellen (Festlegen einer Tabellenbreite, automatischen Zeilenumbruch,...)
            \usepackage{booktabs} % Das Hauptaugenmerk von booktabs liegt dabei auf der Gestaltung der horizontalen Linien innerhalb einer Tabelle.  
            \usepackage{multirow} % Package, dass die Zusammenführung von Zellen in einer Tabelle ermöglicht
            \newcolumntype{L}[1]{>{\RaggedRight\arraybackslash}p{#1}}
            \newcolumntype{C}[1]{>{\Centering\arraybackslash}p{#1}}
            \newcolumntype{R}[1]{>{\RaggedLeft\arraybackslash}p{#1}}
            
            % Bild- & Tabellenunterschriften & -quellen
            \usepackage{caption}
            \captionsetup{format=plain,justification=RaggedRight,singlelinecheck=false}
            \usepackage{capt-of}
            \usepackage[labelfont=bf]{caption}			
            \captionsetup[table]{justification=centerlast}
            
            % Seitenformatierung
            \usepackage[left= 3cm, right= 2cm, bottom= 2.5cm, top= 2.5cm]{geometry} % Seitenabstände
            \usepackage{pdflscape} % Package zur Drehung von Seiten
            
            % Überschriften-Schriftgröße und -art ändern
            \addtokomafont{disposition}{\rmfamily}
            %\setkomafont{chapter}{\rmfamily\LARGE\bfseries} % Nicht notwendig für die Klasse 'scrartcl'
            \setkomafont{section}{\rmfamily\Large\bfseries}
            \setkomafont{subsection}{\rmfamily\large\bfseries}
            
            % Deutsches Datumsformat
            \usepackage{datetime}
            \newdateformat{myformat}{\THEDAY{. }\monthname[\THEMONTH] \THEYEAR}
            
            % Kopf- & Fußzeile
            %\usepackage{fancyhdr}
            %\pagestyle{fancy} % Muss vor \renewcommand{\chaptermark} stehen
            %\fancyhf{} % Bereinigt Kopf- & Fußzeile 
            %\fancyfoot[L]{} % Links
            %\fancyfoot[C]{\thepage} % Mitte
            %\fancyfoot[R]{} % Rechts
            %\renewcommand{\chaptermark}[1]{
            %	\markboth{Kapitel \thechapter{}: #1}{}}
            %\renewcommand{\sectionmark}[1]{
            %	\markright{\thesection{} #1}}
            %\renewcommand{\headrulewidth}{0pt} % Dicke der Trennlinie oben
            
            % Tabellen- und Abbildungsverzeichnis nach der Lehrstuhlvorlage
            \usepackage{tocloft}
            \renewcommand{\cftfigpresnum}{Figure }
            \renewcommand{\cfttabpresnum}{Table }
            
            \renewcommand{\cftfigaftersnum}{:}
            \renewcommand{\cfttabaftersnum}{:}
            
            \setlength{\cftfignumwidth}{3cm}
            \setlength{\cfttabnumwidth}{2,5cm}
            
            \setlength{\cftfigindent}{0,5cm}
            \setlength{\cfttabindent}{0,5cm}
            
            %  Zitation
            \usepackage[authordate,backend=biber, doi=false, isbn=false, footmarkoff]{biblatex-chicago}
            % Hinzufügen der Literatur
            %\addbibresource{Literatur.bib}
            
            
            %%%%%%%% Beginn Dokumentinformationen %%%%%%%%
            
            %========== Dokumentbeginn ==========
            \begin{document}
            
            
            % Titelseite
             \newgeometry{left = 2.6cm, right= 2.6cm, bottom= 2.5cm, top= 2.5cm} % Modifiziert die Seitenabstände für die Titelseite
                            
                            
                            \begin{titlepage}
                            \begin{center}

                            \large{Danaos Management Consultants} \\

                            \vspace{1cm}
                            \noindent\rule{\textwidth}{1.5pt}
                            \vspace{0.3cm}

                            \LARGE{\textbf{ "HYUNDAI SMART" }} \\ % Bei Masterarbeit hier den Text austauschen

                            \large{\today}\\

                            \vspace{1cm}


                            \vspace{0.3cm}
                            \noindent\rule{\textwidth}{1.5pt}
                            \vspace{1cm}
                            \end{center}

                            \noindent\normalsize{\textbf{Vessel Basic Info:}} \hfill \normalsize{} 

                            \vspace{0.5cm}

                            \noindent\normalsize{} \hfill \normalsize{} \\
                            \normalsize{IMO: 9296959 )} \\
                            \normalsize{Vessel Type: Container} \\
                            \normalsize{Deadweight: } \\
                            \normalsize{Beam: } \\
                            \normalsize{Length:  }\\
                            \normalsize{Min Draft: 6 m} \\
                            \normalsize{Max Draft: 12 m} \\
                            \normalsize{Max Speed: Sea speed Ballast . Laden } \\
                            \normalsize{
                            Dates of last two propeller cleanings:	 }\\
                            \normalsize{Dates of last two dry docks:  }\\
                            \normalsize{Voyages the vessel most commonly
                            Performs: }



                            \end{titlepage}

                            \restoregeometry % Stellt Seitenabstände wieder her
                           
            
            % Inhaltsverzeichnis
            \newpage % Erzeugt eine neue Seite
            \pagenumbering{Roman} % Umschalten auf römische Seitenzahlnummerierung 
            \tableofcontents % Erzeugt Inhaltsverzeichnis
            
            % Abbildungsverzeichnis
            \newpage % Erzeugt eine neue Seite
            \renewcommand\listfigurename{List Of Figures}
            \listoffigures
            %\addcontentsline{toc}{section}{List Of Figures}
            
            % Tabellenverzeichnis
            %\newpage % Erzeugt eine neue Seite
            %\renewcommand\listtablename{List of Tables}
            %\listoftables
            %\addcontentsline{toc}{section}{List of Tables}
            
            
            % Haupttext
            \newpage % Erzeugt eine neue Seite
            \pagenumbering{arabic} % Umschalten auf arabische Seitenzahlen
            \setcounter{page}{1} % Setzt die Seitenzahl auf 1 zurück		
            
             \section{Summary Table}
                
                
                begin{table}[H]
                                \centering
                                  \caption{Total Actual FOC \footnotesize{(MT/day)}, Total Predicted FOC \footnotesize{(MT/day)}, Percentage Difference(Actual / Predicted), Mean STW (knt), Mean Draft (m) for each leg}
                                    \label{tab:mae}
                                \resizebox{\textwidth}{!}{    
                                     \begin{tabular}{|c|c|c|c|c|c|}
                                    \hline
                                \multirow{1}{*}{} & \multicolumn{5}{|c|}{\textbf{Info per leg}}
                                
                                \\
                                
                                \cline{2-6}
                                    
                                    
                                  
                                        \textbf{HYUNDAI SMART} 
                                     &  \textbf{Total Act FOC} & \textbf{Total Pred FOC} 
                                     
                                     & \textbf{Perc Diff} & \textbf{Mean STW}
                                     
                                     &  \textbf{Mean Draft} \\
                                     \hline 
                                     
                                     %(templateRowTable)s   
                                    \hline
                                    
                                    \hline 
                                    \textbf{Total Avg } &  \multicolumn{1}{|c|
                                    }{\textbf{301026 (MT/day)}} &
                                    \multicolumn{1}{|c|}{\textbf{302142 (MT/day)}}  & 
                                    \multicolumn{1}{|c|}{\textbf{{0,76 \% }}} &
                                    \multicolumn{1}{|c|}{\textbf{14.8 }(knt)} &
                                    \multicolumn{1}{|c|}{\textbf{8.8 (m)}}\\ 
                                    \hline 
                                  
                                    \end{tabular}
                                }
                                
                                \end{table}
                                \raggedbottom

                            
                
                \newpage
                
                \section{FOC estimation models performance on legs}
                
                
                
            
            % Literaturverzeichnis
            \newpage % Erzeugt eine neue Seite
            \addcontentsline{toc}{section}{Literaturverzeichnis}
            \printbibliography[heading=bibliography]
            
            % Anhang
            \newpage % Erzeugt eine neue Seite
            \addcontentsline{toc}{section}{Anhang}
            
            
            % Eidesstattliche Erklärung (nicht notwendig bei Seminar- und Hausarbeiten)
            \newpage % Erzeugt eine neue Seite
            
            
            
            \end{document}
            %========== Dokumentende ==========
                    